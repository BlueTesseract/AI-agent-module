% Opcje klasy 'iithesis' opisane sa w komentarzach w pliku klasy. Za ich pomoca
% ustawia sie przede wszystkim jezyk i rodzaj (lic/inz/mgr) pracy, oraz czy na
% drugiej stronie pracy ma byc skladany wzor oswiadczenia o autorskim wykonaniu.
\documentclass[declaration,shortabstract,inz]{iithesis}

\usepackage[utf8]{inputenc}

%%%%% DANE DO STRONY TYTUŁOWEJ
% Niezaleznie od jezyka pracy wybranego w opcjach klasy, tytul i streszczenie
% pracy nalezy podac zarowno w jezyku polskim, jak i angielskim.
% Pamietaj o madrym (zgodnym z logicznym rozbiorem zdania oraz estetyka) recznym
% zlamaniu wierszy w temacie pracy, zwlaszcza tego w jezyku pracy. Uzyj do tego
% polecenia \fmlinebreak.
%\polishtitle    {Projekt i implementacja biblioteki ułatwiającej tworzenie\fmlinebreak inteligentnych agentów grających w gry planszowe}
\englishtitle   {Projekt i implementacja biblioteki ułatwiającej tworzenie\fmlinebreak inteligentnych agentów grających w gry planszowe}
%\polishabstract {\ldots}
\englishabstract{\ldots}
% w pracach wielu autorow nazwiska mozna oddzielic poleceniem \and
\author         {Mikołaj Kowalik}
% w przypadku kilku promotorow, lub koniecznosci podania ich afiliacji, linie
% w ponizszym poleceniu mozna zlamac poleceniem \fmlinebreak
\advisor        {dr Paweł Rychlikowski}
%\date          {15.01.2019}                     % Data zlozenia pracy
% Dane do oswiadczenia o autorskim wykonaniu
\transcriptnum {283476}                     % Numer indeksu
%\advisorgen    {dr. Pawła Rychlikowskiego} % Nazwisko promotora w dopelniaczu
%%%%%

%%%%% WLASNE DODATKOWE PAKIETY
%
%\usepackage{graphicx,listings,amsmath,amssymb,amsthm,amsfonts,tikz}
%
%%%%% WŁASNE DEFINICJE I POLECENIA
%
%\theoremstyle{definition} \newtheorem{definition}{Definition}[chapter]
%\theoremstyle{remark} \newtheorem{remark}[definition]{Observation}
%\theoremstyle{plain} \newtheorem{theorem}[definition]{Theorem}
%\theoremstyle{plain} \newtheorem{lemma}[definition]{Lemma}
%\renewcommand \qedsymbol {\ensuremath{\square}}
% ...
%%%%%

\begin{document}

%%%%% POCZĄTEK ZASADNICZEGO TEKSTU PRACY

\chapter{Wprowadzenie}
\chapter{Wprowadzenie Teoretyczne}
\section{Przeszukiwanie po możliwych ruchach w grach planszowych}
\section{Algorytm MinMax}
\section{Algorytm Alpha-Beta}
\section{Algorytm Monte Carlo Tree Search}
\chapter{Biblioteka Ułatwiająca Tworzenie Agentów do Gier Planszowych}
\section{Co wchodzi w skład biblioteki}
\section{Czego biblioteka nie zawiera}
\section{Omówienie szablonu gier planszowych}
\section{Propozycja implementowania gier planszowych z wykorzystaniem biblioteki}
\subsection{Tworzenie gry planszowej}
\subsection{Tworzenie Agenta}
\section{Przykład programu napisanego z wykorzystaniem omawianej biblioteki}
\section{Propozycje dalszego rozwoju biblioteki}
\chapter{Zakończenie}
\ldots

%%%%% BIBLIOGRAFIA

%\begin{thebibliography}{1}
%\bibitem{example} \ldots
%\end{thebibliography}

\end{document}
