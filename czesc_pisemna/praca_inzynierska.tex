% Opcje klasy 'iithesis' opisane sa w komentarzach w pliku klasy. Za ich pomoca
% ustawia sie przede wszystkim jezyk i rodzaj (lic/inz/mgr) pracy, oraz czy na
% drugiej stronie pracy ma byc skladany wzor oswiadczenia o autorskim wykonaniu.
\documentclass[declaration,shortabstract,inz]{iithesis}

\usepackage[utf8]{inputenc}

%%%%% DANE DO STRONY TYTUŁOWEJ
% Niezaleznie od jezyka pracy wybranego w opcjach klasy, tytul i streszczenie
% pracy nalezy podac zarowno w jezyku polskim, jak i angielskim.
% Pamietaj o madrym (zgodnym z logicznym rozbiorem zdania oraz estetyka) recznym
% zlamaniu wierszy w temacie pracy, zwlaszcza tego w jezyku pracy. Uzyj do tego
% polecenia \fmlinebreak.
\polishtitle    {Projekt i implementacja biblioteki ułatwiającej tworzenie\fmlinebreak inteligentnych agentów grających w gry planszowe}
\englishtitle   {Projekt i implementacja biblioteki ułatwiającej tworzenie\fmlinebreak inteligentnych agentów grających w gry planszowe}
\polishabstract {W pracy tej została opisana tytułowa biblioteka oraz minimalne wprowadzenie teoretyczne potrzebne do zrozumienia działania, napisanego na potrzeby pracy, modułu python'owego. Niemniej jednak, głównym celem tej pracy było nauczenie czytelnika z korzystania możliwości oferowanych przez powyższą bibliotekę.}
\englishabstract{\ldots}
% w pracach wielu autorow nazwiska mozna oddzielic poleceniem \and
\author         {Mikołaj Kowalik}
% w przypadku kilku promotorow, lub koniecznosci podania ich afiliacji, linie
% w ponizszym poleceniu mozna zlamac poleceniem \fmlinebreak
\advisor        {dr Paweł Rychlikowski}
%\date          {15.01.2019}                     % Data zlozenia pracy
% Dane do oswiadczenia o autorskim wykonaniu
\transcriptnum {283476}                     % Numer indeksu
%\advisorgen    {dr. Pawła Rychlikowskiego} % Nazwisko promotora w dopelniaczu
%%%%%

%%%%% WLASNE DODATKOWE PAKIETY
%
%\usepackage{graphicx,listings,amsmath,amssymb,amsthm,amsfonts,tikz}
%
\usepackage{listings}
\usepackage{listings} %For code in appendix
\lstset
{ %Formatting for code in appendix
    language=Python,
    basicstyle=\footnotesize,
    numbers=left,
    stepnumber=1,
    showstringspaces=false,
    tabsize=1,
    breaklines=true,
    breakatwhitespace=false,
}
%%%%% WŁASNE DEFINICJE I POLECENIA
%
%\theoremstyle{definition} \newtheorem{definition}{Definition}[chapter]
%\theoremstyle{remark} \newtheorem{remark}[definition]{Observation}
%\theoremstyle{plain} \newtheorem{theorem}[definition]{Theorem}
%\theoremstyle{plain} \newtheorem{lemma}[definition]{Lemma}
%\renewcommand \qedsymbol {\ensuremath{\square}}
% ...
%%%%%

\begin{document}

%%%%% POCZĄTEK ZASADNICZEGO TEKSTU PRACY

\chapter{Wprowadzenie}
Praca ta ma na celu pomoc w zapoznaniu czytelnika z omawianą biblioteką.
Zadbano aby nie zabrakło tutaj podstawowego wprowadzenia teoretycznego oraz praktycznego omówienia przypadków użycia tej biblioteki.

\chapter{Wprowadzenie Teoretyczne}
\section{Przeszukiwanie po możliwych ruchach w grach planszowych}
Jednym z podstawowych problemów gier planszowych jest problem przeszukiwania.
Na tym etapie biblioteka głównie jest skupiona na pomocy programiście poradzenia sobie właśnie z tym problemem.
Na potrzeby tej pracy wybrano trzy algorytmy przeszukiwania: MiniMax, Alpha-Beta oraz Monte Carlo Tree Search, które, od strony praktycznej, omówione są także w tym wprowadzeniu.
Sam wstęp teoretyczny jest przewidziany jak bardzo ogólne omówienie tych algorytmów - dający jedynie niewielkie rozróznienie pomiędzy nimi. Jest ono pisane z punktu widzenia przyszłego użytkownika - programisty.
Szczegółowe omówienie łatwo można znaleźć w wielu pracach - dlatego praca ta zapewnia jedynie ramowe przedstawienie tych metod.
\section{Algorytm MiniMax}
\subsection{Pełny algorytm MiniMax}
Algorytm MiniMax w podstawowej wersji (pełny MiniMax) przegląda wszystkie możliwe ruchy i buduje całe drzewo gry.
Algorytm ten zakłada, że zarówno gracz jak i przeciwnik będą wykonywać najlepsze, dla siebie, ruchy.
Na podstawie tak zbudowanego drzewa gry algorytm wybiera najlepszy możliwy ruch.

Wady:
\begin{itemize}
  \item Wolny w działaniu
  \item Potrzebuje bardzo dużo pamięci
  \item Ze względu na dwie pierwsze wady - raczej nie wykorzystywany
\end{itemize}

Zalety:
\begin{itemize}
  \item Podaje optymalne ruchy
  \item Nie wymaga heurystyk
\end{itemize}
\subsection{Algorytm MiniMax z obcięciami}
Algorytm MiniMax z obcięciami działa tak samo jak pełny minimax - z tą różnicą, że gdy algorytm rozbuduje drzewo gry do pewnej wysokości przestaje rozbudowywać i wywołuje funkcję heurystyczną w celu oceny wygranej

Wady:
\begin{itemize}
  \item Wolny w działaniu, ale szybszy niż pełny MiniMax
  \item Wymaga funkcji oceny stanu gry - czesto stworzenie takiej funkcji jest skomplikowanym zadaniem
\end{itemize}

Zalety:
\begin{itemize}
  \item Można ograniczyć wykorzystanie pamięci przez drzewo gry
  \item Dzięki funckji oceny czas działania powinien być szybszy niż w przypadku pełnego MiniMaxa
\end{itemize}
\section{Algorytm Alpha-Beta}
Algorytm Aplha-Beta to w praktyce zmodyfikowana wersja MiniMaxa - algorytm pomija pewne bezsensowne przeszukiwania.
Z punktu widzenia programisty można z tego algorytmu korzystać jak z MiniMaxa

Wady:
\begin{itemize}
  \item Jeśli nie chcemy rozbudowywać drzewa do samego końca to wymaga funkcji oceny stanu gry
\end{itemize}

Zalety:
\begin{itemize}
  \item Dużo szybszy niż MiniMax (przykład to zaimplementowana na potrzeby pracy gra kółko i krzyżyk)- nadaje się do praktycznego wykorzystania
  \item W porównaniu do MiniMax lepsze zarządzanie pamięcią
\end{itemize}
\section{Algorytm Monte Carlo Tree Search}
Algorytm ten jest zbieżny do MiniMaxa (w praktyce wolno zbiega). Posiada szerokie zastosowanie.
Główną siłą napędową sukcesów tego algorytmów jest przeprowadzanie losowych rozgrywek.
Na podstawie wygranych są podejmowane decyzje, który ruch jest najlepszy.
Z punktu widzenia programisty można z niego korzystać tak samo jak z poprzednich metod - warto jednak znać jego wady i zalety.

Wady:
\begin{itemize}
  \item Istnieją gry w których losowa rozgrywka może trwać bardzo długo
  \item Szczególnie przy majej liczbie symulacji łatwo można otrzymać ruch nieoptymalny
\end{itemize}

Zalety:
\begin{itemize}
  \item Dobre zarządzanie pamięcią
  \item Łatwo ograniczyć czasowo jego wykonanie (przez odpowiednią liczbę symulacji)
  \item Nie wymaga heurystyki - losowe rozgrywki spełniają tę rolę
  \item Sukcesy w złożonych, trudnych grach (np. Go)
\end{itemize}
\section{Podsumowanie}
Powyższe przedstawienie miało charakter bardzo praktyczny i ogólny.
Jeśli czytelnik czuje z tego powodu niedosyt wiedzy lub też z ciekawości chciałby poznać szczegóły tych algorytmów to proszony jest o skorzystanie ze źródeł wymienionych w bibliografi.
\chapter{Omówienie biblioteki do pisania agentów}
\section{Co wchodzi w skład biblioteki}
W skład biblioteki wchodzi:
\begin{itemize}
  \item plik agent\textunderscore module.py - jest to moduł do pythona zawierający najważniejsze funkcje
  \item plik game\textunderscore template.py - jest to plik z szablonem gry planszowej mający na celu ułatwić projektowanie gry.
  \item ox.py, connect4.py, fox\textunderscore game.py, reversi.py - przykłady tworzenia gier planszowych
\end{itemize}
\section{Omówienie szablonu gier planszowych}
Swoją pracę z omawianą biblioteką należy zacząć od zrozumienia pliku game\textunderscore template.py.
Na wstępie warto powiedzieć, że prawidłowe korzystanie z tego pliku polega na skopiowaniu go i poprawnym wyedytowaniu na potrzeby nowej gry.
\subsection{class GameTemplate}
W pliku tym znajduje się klasa GameTemplate jest to najważniejsza część tego pliku.
Jak można zauważyć klasa ta zawiera wiele metod - nazwy tych metod nie powinny być zmieniane, ponieważ agent\textunderscore module.py korzysta z nich w swoich funkcjach.
Klasa ta to abstrakcyjny model gry planszowej - zawierający wszystkie niezbędne funkcje - które powinno się zaimplementować.
\subsection{Przykładowe AI}
W tym samym pliku, poniżej wcześniej wspomnianej klasy - można znaleźć klasę NPCPlayer. Ten przykład jest bardzo prosty - jest to zwykły agent grający losowo.
W tym miejscu warto wspomnieć, że nasz moduł do agentów wykorzystuje pole player\textunderscore id - powinien być to unikatowy id dla agenta - niekoniecznie to musi być, jak w przykładzie, jego pionek.
Niemniej jednak na potrzeby funkcji z agenta to pole powinno być zapewnione.

Na końcu tego pliku widać dwie linijki kodu, które odpowiadają za inicjalizacje oraz uruchomienie gry.
\section{Propozycja implementowania gier planszowych z wykorzystaniem biblioteki}
Dla ułatwienia pisania programu oraz szybkiego zapoznania się z biblioteką warto omówić rutynowe czynności, które występują przy pracy z omawianą biblioteką.
Poniżej omówione są dwie zawsze występujące, rutynowe, czynności.
\subsection{Tworzenie gry planszowej}
Tworzenie gry planszowej przy pomocy tej biblioteki sprowadza się do:
\begin{itemize}
  \item Zmiany nazwy klasy GameTemplate na nazwę gry
  \item implementacji logiki gry w metodach tej klasy (zgodnie z komentarzami)
  \item implementacji interfejsu graficznego gry
  \item tetowania programu
\end{itemize}

\subsection{Tworzenie Agenta}
Tworzenie agenta w abstrakcyjnym ujęciu sprowadza sie do:
\begin{itemize}
  \item zmiany nazwy klasy agenta z pliku game\textunderscore template.py
  \item zapisanie logiki agenta w metodzie make\textunderscore move
\end{itemize}

\section{Przykład oraz omówienie programu napisanego z wykorzystaniem omawianej biblioteki}
\subsection{Kod gry kółko i krzyżyk}
\lstinputlisting{../ox.py}
\subsection{Wytłumaczenie powyższego kodu}
Na początku widać klasę OX. Jest to klasa GameTemplate ze zmienioną nazwą.
Kolejno w liniach 4-9 widać inicjalizację klasy (konstruktor):
\lstinputlisting[firstline=4,lastline=9,firstnumber=4]{../ox.py}
Warto zwrócić uwagę, że na samym początku tworzymy pustą planszę do gry w kółko i krzyżyk.
kropki na naszej planszy oznaczają pola wolne.
Klasa GameBoard przyjmuje jako parametr plansze do gry (w dowolnym formacie).
W przypadku naszej gry jej użycie nie jest konieczne.
Sama ta klasa zapewnia najbardziej podstawowe funkcje do debugowania i wspomagające operacje na planszach typu tablica dwuwymiarowa.

Kolejna metoda klasy OX to get\textunderscore opponent
\lstinputlisting[firstline=11,lastline=15,firstnumber=11]{../ox.py}
Metoda ta ma na celu podanie przeciwnika dla podanego w parametrze gracza.
Jest to funkcja z której korzystają algorytmy takie jak alpha-beta, minimax - stąd potrzeba jej implementacji.

Dalej można znaleźć metody dwie kolejne metody wymagane przez bibliotekę:
\lstinputlisting[firstline=17,lastline=26,firstnumber=11]{../ox.py}

Pierwsza z tych funkcji get\textunderscore allowed\textunderscore moves powinna zwracać wszystkie możliwe ruchy dla danego gracza.
Format odpowiedzi to powinna być tablica ruchów właśiwie w dowolnym formacie, który jest rozumiany przez funkcję apply\textunderscore move.
W naszym wypadku gracz nie ma znaczenia, ponieważ możliwe ruchy to wolne pola na planszy (i nie zależy to od gracza).
Funkcja ta przyjmuje gracza jako argument tylko dla wymagań biblioteki.
Obydwie te funkcje są potrzebne do poprawnego działania algorytmów przeszukiwania.

Ostatnie dwie ważne funkcje to is\textunderscore winner oraz game\textunderscore end.
Pierwsza z nich odpowiada czy gracz podany jako argument jest zwycięzcą gry, druga odpowiada na pytanie czy gra jest zakończona.
Podobnie jak wyżej te funkcje są potrzebne dla wyżej wymienionych algorytmów i implementują logikę właściwą dla konkretnej gry.

Pozostała część kodu to funkcjonalności potrzebne do działania gry, stworzone w sposób opisana w poprzednim punkcie tego rozdziału.

\section{Propozycje dalszego rozwoju biblioteki}
Sam temat pracy jest bardzo rozległy.
Praca zaś nie pokrywa w całości tematu pracy - dlatego też zamieszczam tutaj propozycje dalszego rozwoju:
\begin{itemize}
  \item Dodanie algorytmów do wyszukiwania ścieżek (Dijkstra (A*) itp.)
  \item Dodanie innych algorytmów do przeszukiwania (Simulated annealing, Local Beam Search itp.)
  \item Przemyślenia wsparcia dla sieci neuronowych i deep learningu.
  \item Liczne optymalizacje
  \item Ze względu na to, że sam python działa stosunkowo wolno - stworzenie wersji blioteki w C/C++
  \item Dodanie testów automatycznych.
\end{itemize}

\chapter{Zakończenie}

Jak widać, praca nie wyczerpuje w pełni, szerokiego zagadnienia - jakim są algorytmy do gier planszowych.
Praca ta implementuje funkcjonalności, które na poziomie projektowania okazały się najbardziej znaczące.
Niemniej jednak aktualna praca pozwala na wykorzystywanie jej w wielu prostszych grach, a wymuszona architekura ma pomóc pozbyć się niepotrzebnych, powtarzalnych problemów związanych z projektowaniem kodu w grach planszowych.

%%%%% BIBLIOGRAFIA

%\begin{thebibliography}{1}
%\bibitem{example} \ldots
%\end{thebibliography}

\end{document}
